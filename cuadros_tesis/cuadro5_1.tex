\documentclass[border=3mm,preview]{standalone}
\usepackage[]{graphicx}\usepackage[]{color}

\usepackage{alltt}\usepackage[]{graphicx}\usepackage[]{color}
\usepackage[utf8]{inputenc}
\usepackage[spanish]{babel}
\usepackage{fancyhdr}
\usepackage{lastpage}
\usepackage{lscape} %Para seleccionar páginas horizontales 
\usepackage[cc]{titlepic} 
\usepackage[hidelinks]{hyperref}  %Para links ocultos
\usepackage{url} %Para direcciones web
\usepackage[makeroom]{cancel}
\pagestyle{fancy}
\usepackage{multicol} % muchas columnas
\usepackage{booktabs} % midrule, bottomrule
\usepackage{titlesec} % cambiar secuencias de TOC 
\usepackage{siunitx}
\usepackage{varwidth} % PAra los documentos standalone y el ajuste preciso
\usepackage{subcaption} % Para usar subcaptions en las matrices de imágenes 
\sisetup{binary-units = true,table-format=7.0}
\let\bold\boldsymbol
\let\bf\mathbf


\fancyhf{}
 \setcounter{page}{1}
%\rfoot{Página \thepage \hspace{1pt} de \pageref{LastPage}}
\setcounter{section}{0}

\usepackage{blindtext}  %Texo sin sentido
\usepackage{amsmath} \newenvironment{smatrix}{\left(\begin{smallmatrix}}{\end{smallmatrix}\right)} %SMALL
\usepackage{amsthm}
\usepackage{amsfonts}
\DeclareMathOperator{\sgn}{sgn} %Para formalizar la función signo 
\usepackage{enumerate}
\usepackage{dsfont} %Para usar una indicadora
\numberwithin{equation}{section}
\usepackage{xcolor}
\usepackage[backend=bibtex]{biblatex}
\bibliography{biblio.bib}
\usepackage{booktabs,caption}
\usepackage[flushleft]{threeparttable}
\begin{document}
\centering
\begin{varwidth}{\linewidth}
\centering
\begin{tabular}{@{}lllllllll@{}}
\toprule
\multicolumn{1}{l}{} & \multicolumn{4}{c}{Estadístico} &
\multicolumn{3}{c}{Valores críticos} \\
\cmidrule(l){2-5} \cmidrule(l){4-8} \\
\multicolumn{1}{l}{} & \multicolumn{2}{c}{Palta} & \multicolumn{2}{c}{Tomate} & 
\multicolumn{3}{c}{} \\
\multicolumn{1}{l}{$\mathcal{H}_0$} & \multicolumn{1}{c}{Mayorista$^{a}$} &
 \multicolumn{1}{c}{Supermercado$^{b}$} &
  \multicolumn{1}{c}{Mayorista$^{a}$} &
 \multicolumn{1}{c}{Supermercado$^{b}$} &
\multicolumn{1}{l}{90\%}&
\multicolumn{1}{l}{95\%}&
\multicolumn{1}{l}{99\%}
\\
\midrule
$\tau_{2} $  & -2.5725 &  -1.6393 & & & -2.57 & -2.87 & -3.44 \\
$\phi_{1} $  & 3.4095  &  1.6354 & & & 3.79 & 4.61 &  6.47\\
\bottomrule
\end{tabular}
\begin{tablenotes}
\small 
\item $^{a}$: Con un rezago, de acuerdo a criterio BIC. 
\item $^{b}$: Con un rezago, de acuerdo a criterio BIC. 
\end{tablenotes}
\end{varwidth}
\end{document}