\documentclass[12pt, twoside]{book}\usepackage[]{graphicx}\usepackage[]{color}

\usepackage{alltt}\usepackage[]{graphicx}\usepackage[]{color}
\usepackage[utf8]{inputenc}
\usepackage[spanish]{babel}
\usepackage{fancyhdr}
\usepackage{lastpage}
\usepackage{lscape} %Para seleccionar páginas horizontales 
\usepackage[cc]{titlepic} 
\usepackage[hidelinks]{hyperref}  %Para links ocultos
\usepackage{url} %Para direcciones web
\usepackage[makeroom]{cancel}
\pagestyle{fancy}
\usepackage{multicol} % muchas columnas
\usepackage{booktabs} % midrule, bottomrule
\usepackage{titlesec} % cambiar secuencias de TOC 
\let\bold\boldsymbol
\let\bf\mathbf


\fancyhf{}
 \setcounter{page}{1}
%\rfoot{Página \thepage \hspace{1pt} de \pageref{LastPage}}
\setcounter{section}{0}

\usepackage[caption=false]{subfig}
\usepackage{blindtext}  %Texo sin sentido
\usepackage{amsmath} \newenvironment{smatrix}{\left(\begin{smallmatrix}}{\end{smallmatrix}\right)} %SMALL
\usepackage{amsthm}
\usepackage{amsfonts}
\DeclareMathOperator{\sgn}{sgn} %Para formalizar la función signo 
\usepackage{enumerate}
\usepackage{dsfont} %Para usar una indicadora
\numberwithin{equation}{section}
\usepackage{xcolor}
\usepackage[backend=bibtex]{biblatex}
\bibliography{biblio.bib}
\usepackage{booktabs,caption}
\usepackage[flushleft]{threeparttable}

\usepackage{mdframed} %Para usar recuadros

    \usepackage{framed}

    \colorlet{shadecolor}{blue!15}

    \newtheorem{theorem}{Observación}
    \newenvironment{theo}
      {\begin{shaded}\begin{theorem}}
      {\end{theorem}\end{shaded}}
      \numberwithin{theorem}{section}

\colorlet{shadecolor}{red!15}

    \newtheorem{teorema}{Proposición}
    \newenvironment{teo}
      {\begin{shaded}\begin{teorema}}
      {\end{teorema}\end{shaded}}
      \numberwithin{teorema}{section}

\colorlet{shadecolor}{gray!15}
    \newtheorem{defi}{Definición}
    \newenvironment{defin}
      {\begin{shaded}\begin{defi}}
      {\end{defi}\end{shaded}}
      \numberwithin{defi}{section}
%\newtheorem{rexample}{Código R}[subsection]
\newtheorem{prop}{Proposición}
%\newtheorem{defi}{Definición}
\numberwithin{prop}{section}
\numberwithin{defi}{section}
\theoremstyle{plain}
\setlength{\textfloatsep}{10pt}
\usepackage{multicol}
\usepackage{float}

\usepackage[affil-it]{authblk}
\usepackage{setspace}
\usepackage{listings}

\usepackage{geometry}
\geometry{a4paper, left=3cm, right=3cm, top=3cm, bottom=3cm}
\newcommand*\rfrac[2]{{}^{#1}\!/_{#2}}
\title{Análisis de datos: Transmisión de precios}
\author{H\'ector Garrido Henr\'iquez\thanks{Ingeniero Comercial. Contacto: \texttt{hectorgarridohenriquez@gmail.com}} \\ 
Profesor: Sergio Contreras Espinoza}

\affil{Mag\'ister en Matem\'atica Menci\'on Estad\'istica \\ Universidad del B\'io-B\'io}
\IfFileExists{upquote.sty}{\usepackage{upquote}}{}

\allowdisplaybreaks
\IfFileExists{upquote.sty}{\usepackage{upquote}}{}
\IfFileExists{upquote.sty}{\usepackage{upquote}}{}
\begin{document}
%\SweaveOpts{concordance=TRUE}
\begin{titlepage}
\begin{center}
\includegraphics[scale=0.07]{./figure/logo.png}\\
\textsc{\Large Universidad del Bío-Bío \\[0.5cm] Facultad de Ciencias}\\[1cm] % University name
\textsc{\Large}\\[0.3cm] % Thesis type

\noindent\makebox[\linewidth]{\rule{\textwidth}{1pt}} 
{\huge Transmisi\'on Asim\'etrica de Precios en el sector de la palta en Chile:\\[0.3cm] Evidencia desde un modelo TVECM}\\[0.4cm] % Thesis title
\noindent\makebox[\linewidth]{\rule{\textwidth}{1pt}} 

\textsc{\Large}\\[0.5cm] % Thesis type

\begin{minipage}{0.45\textwidth}
\begin{flushleft} \large
\emph{Autor:}\\
Héctor Garrido Henríquez % Author name - remove the \href bracket to remove the link
\end{flushleft}
\end{minipage}
\begin{minipage}{0.45\textwidth}
\begin{flushleft} \large
\emph{Profesor(es) Guía(s):} \\
Dr. Sergio Contreras Espinoza \\ Dra. Monia Ben Kaabia 
\end{flushleft}
\end{minipage}\\[2cm]
 
\large \textit{Tesis para optar al grado de Magíster en Matemática con mención en Estadística}\\[0.3cm] % University requirement text
\textit{}\\[0.4cm]
\ Departamento de Estadística \\
[0.4cm]\ Departamento de Matemática
\\[1cm] % Research group name and department name
 
{\large \today}\\[2cm] % Date
%\includegraphics{Logo} % University/department logo - uncomment to place it
 
%\vfill
\end{center}

\end{titlepage}
\newpage



\tableofcontents

\listoffigures
\listoftables
\onehalfspacing
\chapter*{Agradecimientos}
\chapter*{Abstract}
\chapter{Introducción}
\section{introducción}
\section{justificación del trabajo}



\chapter{Análisis univariante de las series de precios}
\section{Fuentes de información}

\subsection{Imputación de valores perdidos}




\begin{knitrout}
\definecolor{shadecolor}{rgb}{0.969, 0.969, 0.969}\color{fgcolor}\begin{kframe}


{\ttfamily\noindent\bfseries\color{errorcolor}{\#\# Error in file(filename, "{}r"{}, encoding = encoding): no se puede abrir la conexión}}\end{kframe}
\end{knitrout}


\begin{knitrout}
\definecolor{shadecolor}{rgb}{0.969, 0.969, 0.969}\color{fgcolor}\begin{kframe}


{\ttfamily\noindent\bfseries\color{errorcolor}{\#\# Error in as.ts(x): objeto 'precio\_mayorista' no encontrado}}

{\ttfamily\noindent\bfseries\color{errorcolor}{\#\# Error in KalmanRun(precio\_mayorista, fit1\$model): objeto 'precio\_mayorista' no encontrado}}

{\ttfamily\noindent\bfseries\color{errorcolor}{\#\# Error in which(is.na(precio\_mayorista)): objeto 'precio\_mayorista' no encontrado}}

{\ttfamily\noindent\bfseries\color{errorcolor}{\#\# Error in eval(expr, envir, enclos): objeto 'precio\_mayorista' no encontrado}}

{\ttfamily\noindent\bfseries\color{errorcolor}{\#\# Error in eval(expr, envir, enclos): objeto 'id.na1' no encontrado}}

{\ttfamily\noindent\bfseries\color{errorcolor}{\#\# Error in eval(expr, envir, enclos): objeto 'y1' no encontrado}}

{\ttfamily\noindent\bfseries\color{errorcolor}{\#\# Error in as.ts(x): objeto 'precio\_supermercado' no encontrado}}

{\ttfamily\noindent\bfseries\color{errorcolor}{\#\# Error in KalmanRun(precio\_supermercado, fit2\$model): objeto 'precio\_supermercado' no encontrado}}

{\ttfamily\noindent\bfseries\color{errorcolor}{\#\# Error in which(is.na(precio\_supermercado)): objeto 'precio\_supermercado' no encontrado}}

{\ttfamily\noindent\bfseries\color{errorcolor}{\#\# Error in eval(expr, envir, enclos): objeto 'precio\_supermercado' no encontrado}}

{\ttfamily\noindent\bfseries\color{errorcolor}{\#\# Error in eval(expr, envir, enclos): objeto 'id.na2' no encontrado}}

{\ttfamily\noindent\bfseries\color{errorcolor}{\#\# Error in eval(expr, envir, enclos): objeto 'y2' no encontrado}}\end{kframe}
\end{knitrout}

\begin{knitrout}
\definecolor{shadecolor}{rgb}{0.969, 0.969, 0.969}\color{fgcolor}\begin{kframe}


{\ttfamily\noindent\bfseries\color{errorcolor}{\#\# Error in ts.plot(exp(precio\_mayorista), exp(precio\_supermercado), lty = 1:2, : objeto 'precio\_mayorista' no encontrado}}

{\ttfamily\noindent\bfseries\color{errorcolor}{\#\# Error in strwidth(legend, units = "{}user"{}, cex = cex, font = text.font): plot.new has not been called yet}}\end{kframe}
\end{knitrout}


\begin{knitrout}
\definecolor{shadecolor}{rgb}{0.969, 0.969, 0.969}\color{fgcolor}\begin{kframe}


{\ttfamily\noindent\bfseries\color{errorcolor}{\#\# Error in plot(precio\_mayorista, lty = 2, ylab = "{}Precio tomate mayorista (\$/kilo)"{}, : objeto 'precio\_mayorista' no encontrado}}

{\ttfamily\noindent\bfseries\color{errorcolor}{\#\# Error in lines(y1, lwd = 1, lty = 1): objeto 'y1' no encontrado}}

{\ttfamily\noindent\bfseries\color{errorcolor}{\#\# Error in time(y1): objeto 'y1' no encontrado}}

{\ttfamily\noindent\bfseries\color{errorcolor}{\#\# Error in strwidth(legend, units = "{}user"{}, cex = cex, font = text.font): plot.new has not been called yet}}

{\ttfamily\noindent\bfseries\color{errorcolor}{\#\# Error in plot(precio\_supermercado, lty = 2, ylab = "{}Precio tomate supermercado (\$/kilo)"{}, : objeto 'precio\_supermercado' no encontrado}}

{\ttfamily\noindent\bfseries\color{errorcolor}{\#\# Error in lines(y2, lwd = 1, lty = 1): objeto 'y2' no encontrado}}

{\ttfamily\noindent\bfseries\color{errorcolor}{\#\# Error in time(y2): objeto 'y2' no encontrado}}

{\ttfamily\noindent\bfseries\color{errorcolor}{\#\# Error in strwidth(legend, units = "{}user"{}, cex = cex, font = text.font): plot.new has not been called yet}}\end{kframe}
\end{knitrout}


\section{Análisis del orden de integración de las series}
\subsection{Análisis gráfico}

\begin{knitrout}
\definecolor{shadecolor}{rgb}{0.969, 0.969, 0.969}\color{fgcolor}\begin{kframe}


{\ttfamily\noindent\bfseries\color{errorcolor}{\#\# Error in plot(y1, main = "{}a) Evolución log(precios) mayoristas, 2008-2016"{}, : objeto 'y1' no encontrado}}

{\ttfamily\noindent\bfseries\color{errorcolor}{\#\# Error in NCOL(x): objeto 'y1' no encontrado}}

{\ttfamily\noindent\bfseries\color{errorcolor}{\#\# Error in NCOL(x): objeto 'y1' no encontrado}}

{\ttfamily\noindent\bfseries\color{errorcolor}{\#\# Error in eval(predvars, data, env): objeto 'y1' no encontrado}}\end{kframe}
\end{knitrout}


\begin{knitrout}
\definecolor{shadecolor}{rgb}{0.969, 0.969, 0.969}\color{fgcolor}\begin{kframe}


{\ttfamily\noindent\bfseries\color{errorcolor}{\#\# Error in plot(y2, main = "{}a) Evolución log(precios) supermercado, 2008-2016"{}, : objeto 'y2' no encontrado}}

{\ttfamily\noindent\bfseries\color{errorcolor}{\#\# Error in NCOL(x): objeto 'y2' no encontrado}}

{\ttfamily\noindent\bfseries\color{errorcolor}{\#\# Error in NCOL(x): objeto 'y2' no encontrado}}

{\ttfamily\noindent\bfseries\color{errorcolor}{\#\# Error in eval(predvars, data, env): objeto 'y2' no encontrado}}\end{kframe}
\end{knitrout}

\section{Contrastes de raíz unitaria}
\subsection{Contraste de Dickey-Fuller Aumentado}

El contraste más utilizado en la investigación aplicada, dada su simplicidad, es el contraste propuesto por \cite{fuller1976} y \cite{dickey1981}. Para aplicar este contraste existen dos posibles modelos 

Si $y_{t}$ satisface la siguiente ecuación

\begin{equation}
y_{t} = \alpha+\rho y_{t-1}+\epsilon_{t}\qquad (t=1,...,n)
\end{equation}
Donde $\epsilon_{t}\sim \mathcal{N}(0,\sigma^{2})$. 

Si $y_{t}$ satisface la siguiente ecuación 

Como puede observarse en el cuadro \ref{tab-1}, existen 3 estadísticos, $\Phi_{1},\quad \Phi_{2}$ y $\Phi_{3}$, y sus respectivas hipótesis que pueden ser utilizados. Mientras $\Phi_{1}$
\begin{equation}
y_{t} = \alpha+\beta\left(t-1-\frac{1}{2}n\right)+\rho y_{t-1}+\epsilon_{t}\qquad (t=1,...,n)
\end{equation}
Donde $\epsilon_{t}\sim \mathcal{N}(0,\sigma^{2})$. 

\begin{table}[h]
\centering
\begin{threeparttable}
\caption{Hipótesis del contraste de Dickey-Fuller}
\begin{tabular}{@{}llrllll@{}}
\toprule
\multicolumn{2}{l}{Estadístico} & \multicolumn{2}{c}{$\mathcal{H}_{0}$} &
\multicolumn{2}{c}{$\mathcal{H}_{a}$} \\
\cmidrule(l){3-4} \cmidrule(l){5-6} \\
\multicolumn{2}{l}{$\Phi_{1}$} &
\multicolumn{2}{l}{$(\alpha,\rho)=(0,1)$} &
\multicolumn{2}{l}{$(\alpha,\rho)\neq(0,1)$} \\
\multicolumn{2}{l}{$\Phi_{2}$} &
\multicolumn{2}{l}{$(\alpha,\beta, \rho)=(0,0,1)$} &
\multicolumn{2}{l}{$(\alpha,\beta,\rho)\neq(0,0,1)$} \\
\multicolumn{2}{l}{$\Phi_{3}$} &
\multicolumn{2}{l}{$(\alpha,\beta, \rho)=(\alpha,0,1)$} &
\multicolumn{2}{l}{$(\alpha,\beta,\rho)\neq(\alpha,0,1)$} \\
\bottomrule
\end{tabular}
\label{tab-1}
\begin{tablenotes}
\small
\item Fuente: Elaboración propia basado en Dickey y Fuller (1981)
\end{tablenotes}
\end{threeparttable}
\end{table}

\subsection{Contraste de Phillips Perron (1992)}

De manera similar al contraste anterior \cite{phillips1988} proponen 
\subsection{Contraste Kiatkowsky, Pesaran, Schmidt \& Shin (1992)}
\subsection{Contraste de Elliot, Rothenberg \& Stock (1993)} 

\subsection{Contraste de Canova \& Hansen (1995) para estacionalidad estable}

\section{Resultados de los contrastes de raíz unitaria}

\begin{knitrout}
\definecolor{shadecolor}{rgb}{0.969, 0.969, 0.969}\color{fgcolor}\begin{kframe}


{\ttfamily\noindent\bfseries\color{errorcolor}{\#\# Error in as.matrix(y): objeto 'y1' no encontrado}}

{\ttfamily\noindent\bfseries\color{errorcolor}{\#\# Error in as.matrix(y): objeto 'y2' no encontrado}}

{\ttfamily\noindent\bfseries\color{errorcolor}{\#\# Error in as.matrix(y): objeto 'y1' no encontrado}}

{\ttfamily\noindent\bfseries\color{errorcolor}{\#\# Error in as.matrix(y): objeto 'y2' no encontrado}}\end{kframe}
\end{knitrout}


\begin{table}[H]
\centering
\begin{threeparttable}
\caption{Contraste de Dickey-Fuller aumentado (con tendencia determinista)}
\begin{tabular}{@{}lrllll@{}}
\toprule
\multicolumn{1}{l}{} & \multicolumn{2}{c}{Estadístico} &
\multicolumn{3}{c}{Valores críticos} \\
\cmidrule(l){2-3} \cmidrule(l){4-6} \\
\multicolumn{1}{l}{$\mathcal{H}_0$} & \multicolumn{1}{c}{Mayorista$^{a}$} &
 \multicolumn{1}{c}{Supermercado$^{b}$} &
\multicolumn{1}{l}{90\%}&
\multicolumn{1}{l}{95\%}&
\multicolumn{1}{l}{99\%}
\\
\midrule
$\tau_{3} $  & -3.1822 &  -2.0437 & -3.13 & -3.42 & -3.98 \\
$\phi_{2} $  & 3.393   &  1.5448 & 4.05 & 4.71 & 6.15 \\
$\phi_{3}$   & 5.0686  &  2.0907 &    5.36    &   6.30   &  8.34    \\ 
\bottomrule
\end{tabular}
\label{tab-2}
\begin{tablenotes}
\small 
\item $^{a}$: Con un rezago, de acuerdo a criterio BIC. 
\item $^{b}$: Con un rezago, de acuerdo a criterio BIC. 
\end{tablenotes}
\end{threeparttable}
\end{table}

\begin{table}[H]
\centering
\begin{threeparttable}
\caption{Contraste de Dickey-Fuller aumentado (con drift)}
\begin{tabular}{@{}lrllll@{}}
\toprule
\multicolumn{1}{l}{} & \multicolumn{2}{c}{Estadístico} &
\multicolumn{3}{c}{Valores críticos} \\
\cmidrule(l){2-3} \cmidrule(l){4-6} \\
\multicolumn{1}{l}{$\mathcal{H}_0$} & \multicolumn{1}{c}{Mayorista$^{a}$} &
 \multicolumn{1}{c}{Supermercado$^{b}$} &
\multicolumn{1}{l}{90\%}&
\multicolumn{1}{l}{95\%}&
\multicolumn{1}{l}{99\%}
\\
\midrule
$\tau_{2} $  & -2.7628 &  -1.7114 & -2.57 & -2.87 & -3.44 \\
$\phi_{1} $  & 3.8373  &  1.6909 & 1.6909 & 4.61 & 3.79 \\
\bottomrule
\end{tabular}
\label{tab-3}
\begin{tablenotes}
\small 
\item $^{a}$: Con un rezago, de acuerdo a criterio BIC. 
\item $^{b}$: Con un rezago, de acuerdo a criterio BIC. 
\end{tablenotes}
\end{threeparttable}
\end{table}







































