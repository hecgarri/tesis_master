\chapter{Marco Teórico}

\section{Transmisión de precios: Transmisión Vertical y Horizonal}
\section{Asimetrías en la transmisión de precios}

Si la transmisión de precios es perfecta, puede afirmarse que la elasticidad de los precios al consumidor ($P_{t}^{c}$) con respecto a los precios al productor ($P_{t}^{p}$) debe ser unitaria. Es decir: 

\begin{align}
\log P_{t}^{c} & = \beta_{0}+\beta_{1} \log(P_{t}^{p}) \\ 
P_{t}^{c} & = \exp\{\beta_{0}+ \log([P_{t}^{p}]^{\beta_{1}})\} \\
P_{t}^{c} & = \exp\{\beta_{0}\}[P_{t}^{P}]^{\beta_1} \label{margen}
\end{align}

Tal y como señalan Tiffin \& Dawson (2001), si $\beta_{1}=1$, entonces el margen comercial se obtiene como $e^{\beta0}-1$

Meyer, J., \& Cramon-Taubadel, S. (2004). Asymmetric price transmission: a survey. Journal of agricultural economics, 55(3), 581-611.

\section{Razones económicas para el origen de asimetrías en el proceso de transmisión de precios}
\subsection{Poder de mercado}
\subsection{Costos de Menú}
\subsection{Razones institucionales: Regulación del mercado}
\subsection{Costos de transporte}
\subsection{Inflación}
\subsection{Costos de búsqueda}
\subsection{Prácticas en la gestión de inventarios}

\section{Clasificación de los tipos de asimetrías y diferentes estrategias empíricas utilizadas}

Incluir principalmente las ideas del trabajo de Frey y Manera e incluir algunos desarrollos recientes sobre la materia, como bien pueden ser los estudios sobre transmisión de la volatilidad. 
\section{Evidencia Internacional}
\section{Evidencia Nacional}
